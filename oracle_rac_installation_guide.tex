\documentclass{article}
\usepackage{graphicx}
\usepackage[T1]{fontenc}
\usepackage[spanish]{babel}
\graphicspath{ {./resources/} }
\usepackage{float}

\begin{document}

\section{Introducción}

La siguiente guia de instalación busca guiar al usuario en la configuración básica de un laboratorio de Oracle Real Application Clusters, esta guia no busca enseñarle al usuario a utilizar Linux pero si buscara proporcionar un poco de contexto adicional sobre ciertos comandos de Unix. El laboratorio utiliza 3 maquinas virtuales, las cuales usaran Linux con distribuciones basadas en Ubuntu y RHEL, por otro lado el sistema operativo puede ser cualquiera de preferencia por el usuario, pero en este caso se usara Pop\_Os!, el cual se encuentra basado en Ubuntu.

\section{Configuración de Maquinas Virtuales}

El primer paso consiste en preparar nuestras maquinas virtuales para los nodos de Oracle DB, estos usaran Oracle Linux 7 con Oracle DB 19c, aca cada nodo tiene cuenta con los siguientes requerimientos.

\begin{itemize}
    \item 4 GB de memoria RAM.
    \item 100 GB de almacenamiento físico.
    \item 2 núcleos de procesamiento.
\end{itemize}

Empezando con la creación del primer nodo en este cluster lo llamaremos "node1", aparte de las especificaciones ya mencionadas es necesario crear 3 adaptadores de red:

\begin{itemize}
    \item “Host-only Adapter": Usado para conectar a la base de datos hacia otros aplicativos. 
    \item "Internal Network Adapter": Usado por la red interna del clúster. 
    \item "Bridged Adapter": Usado para conectarse hacia el internet, usando al sistema operativo huésped. 
\end{itemize}

\begin{figure}[H]
    \begin{center}
        \includegraphics[width=0.95\textwidth]{vm_base.png}
    \end{center}
    \caption{Pantalla de inicio de Configuración de la Máquina Virtual}
\end{figure}

Test

\end{document}
