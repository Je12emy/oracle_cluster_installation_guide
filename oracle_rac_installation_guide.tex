\documentclass{article}
\usepackage{graphicx}
\usepackage[T1]{fontenc}
% LTeX: language=es
% LTeX: language=en
\usepackage[spanish]{babel}
\graphicspath{ {./resources/} }
\usepackage{float}
\usepackage{listings}
\usepackage{xcolor}

\definecolor{codegreen}{rgb}{0,0.6,0}
\definecolor{codegray}{rgb}{0.5,0.5,0.5}
\definecolor{codepurple}{rgb}{0.58,0,0.82}
\definecolor{backcolour}{rgb}{0.95,0.95,0.92}

\lstdefinestyle{mystyle}{
    backgroundcolor=\color{backcolour},   
    commentstyle=\color{codegreen},
    keywordstyle=\color{magenta},
    numberstyle=\tiny\color{codegray},
    stringstyle=\color{codepurple},
    basicstyle=\ttfamily\footnotesize,
    breakatwhitespace=false,         
    breaklines=true,                 
    captionpos=b,                    
    keepspaces=true,                 
    showspaces=false,                
    showstringspaces=false,
    showtabs=false,                  
    tabsize=2
}

\begin{document}

\section{Introducción}

La siguiente guía de instalación busca guiar al usuario en la configuración básica de un laboratorio de Oracle Real Application Clusters, esta guía no busca enseñarle al usuario a utilizar Linux, pero si buscara proporcionar un poco de contexto adicional sobre ciertos comandos de Unix. El laboratorio utiliza 3 máquinas virtuales, las cuales usaran Linux con distribuciones basadas en Ubuntu y RHEL, por otro lado el sistema operativo puede ser cualquiera de preferencia por el usuario, pero en este caso se usará Pop\_Os!, el cual se encuentra basado en Ubuntu.

\section{Configuración de Máquinas Virtuales}

El primer paso consiste en preparar nuestras máquinas virtuales para los nodos de Oracle DB, estos usarán Oracle Linux 7 con Oracle DB 19c, acá cada nodo tiene cuenta con los siguientes requerimientos.

\begin{itemize}
	\item 4 GB de memoria RAM.
	\item 100 GB de almacenamiento físico.
	\item 2 núcleos de procesamiento.
\end{itemize}

Empezando con la creación del primer nodo en este cluster sera llamado ``node1'', aparte de las especificaciones ya mencionadas es necesario crear 3 adaptadores de red:

\begin{itemize}
	\item ``Host-only Adapter'': Usado para conectar à la base de datos hacia otros aplicativos.
	\item ``Internal Network Adapter'': Usado por la red interna del clúster.
	\item ``Bridged Adapter'': Usado para conectarse hacia el internet, usando al sistema operativo huésped.
\end{itemize}

\begin{figure}[H]
	\begin{center}
		\includegraphics[width=0.95\textwidth]{vm_base.png}
	\end{center}
	\caption{Pantalla de inicio de Configuración de la Máquina Virtual}
\end{figure}

Al momento de iniciar la máquina virtual e insertar el archivo ISO con Oracle Linux 7, podemos empezar a configurar la instalación del sistema operativo. Acá iniciaremos con la configuración de partición del disco desde acá se selecciona ``Installation Destination'', en donde se puede ver el disco creado y las opciones de particiones, acá se seleccionará ``I will configure Partitioning''. Acá crearemos las siguientes particiones.

\begin{center}
	\begin{tabular}{ |c|c|c| }
		\hline
		\multicolumn{2}{|c|}{Lista de Particiones}    \\
		\hline
		Nombre de Partición & Capacidad               \\
		\hline
		/boot               & 2 GB de almacenamiento  \\
		/root               & 5 GB de almacenamiento  \\
		/ o root            & 50 GB de almacenamiento \\
		/swap               & 8 GB de almacenamiento  \\
		\hline
	\end{tabular}
\end{center}

Al volver al menú de inicio de nuestro instalador, se selecciona el software por instalar en nuestra nueva instalación, acá se selecciona la opción ``Software Selection'', acá se puede escoger entre varios ambientes, pero se seleccionará ``Server with GUI'' con los siguientes paquetes de software.

\begin{itemize}
	\item ``Hardware Monitoring Utilities''
	\item ``Large Systems Performance''
	\item ``Network file system client''
	\item ``Performance Tools''
	\item ``Compatibility Libraries''
	\item ``Development Tools''
\end{itemize}

Al volver al inicio del instalador se puede configurar las opciones de red en ``Network and Hostname'' para configurar los adaptadores de red instalados en las máquinas virtuales. 

Después de habilitar el primer adaptador de red, seleccione ``Configure'' y en la pestaña de ``ipv4'' se usara la dirección IP ``192.168.24.1/24'' con una salida de ``0.0.0.0''  manualmente.

\begin{figure}[H]
	\begin{center}
		\includegraphics[width=0.95\textwidth]{vm_networking.png}
	\end{center}
	\caption{Configuración de Red de la Instalación de Oracle Linux}
\end{figure}

Se aplicará esta misma configuración manual en el siguiente adaptador disponible, pero esta vez con una dirección IP de ``192.168.10.1/24'' con una salida de ``0.0.0.0'', finalmente para el último adaptador habilitaremos direccionamiento automático con DHCP. Se puede continuar con la instalación del sistema operativo, en la siguiente pantalla podemos seleccionar una contraseña para el usuario ``sudo'', para este ejercicio se usará ``root'' como contraseña.

\section{Configuración del Sistema Operativo}

Una vez en el ambiente de escritorio es necesario instalar software mediante el gestor de paquetes, en este caso esta distribución se encuentra basada en RHEL así que el gestor de paquete es yum.

\begin{lstlisting}[style=mystyle,language=bash]
	$ sudo yum update
	$ sudo yum install -y oracle-database-preinstall-19c.x86_64
	$ sudo yum install oracleasm-support
	$ sudo yum install bind* --skip-broken
\end{lstlisting}

\end{document}
